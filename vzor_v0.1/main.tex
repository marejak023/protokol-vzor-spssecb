% AUTOR: JAKUB MAREŠ
% VERZE: v0.1
% LICENCE: MIT
% POZNÁMKA: Prosím ponechte dole watermark v podobě orámečkovaného textu, kde tuto předlohu stáhnou. Pomůže to v šíření této předlohy. DÍKY! Pokud máte nějaké připomínky nebo vám něco nesedí, tak mě prosím kontaktuje.
% ZDROJ: https://github.com/marejak023

\documentclass[a4paper, 12pt]{article}
\usepackage[utf8]{inputenc}
\usepackage[czech]{babel}
\usepackage{parskip}
\usepackage[margin=2.54cm,footskip=0.25in]{geometry}
\usepackage{enumitem}
\usepackage{graphicx}
\usepackage{circuitikz}
\usepackage{pgfplots}
\usepackage{tcolorbox}

\setcounter{section}{-1}
\pagenumbering{gobble}
\setcounter{section}{-1}
\setlength{\parskip}{2em}
\newcommand{\fakesection}[1]{%
\par\refstepcounter{section}% Increase section counter
\sectionmark{#1}% Add section mark (header)
\addcontentsline{toc}{section}{\protect\numberline{\thesection}#1}% Add section to ToC
	% Add more content here, if needed.
	% ToC při protokolech nepoužíváme
}
\setlength{\parindent}{0in} % Vypne odsazení u nového odstavce
\pgfplotsset{compat=1.3} % mění vzdálenost pro y-label u pgfplots

\begin{document}

\begin{center}
\begin{table}[h]
\hspace{-0.78cm}
    \begin{tabular}{|p{3cm}|p{3cm}|p{3cm}|p{3cm}|p{3cm}|}
        \hline
        \multicolumn{5}{|c|}{\begin{tabular}[c]{@{}c@{}}STŘEDNÍ PRŮMYSLOVÁ ŠKOLA V ČESKÝCH BUDĚJOVICÍCH, DUKELSKÁ 13\\ \textbf{PROTOKOL}\\ \textbf{O LABORATORNÍM CVIČENÍ}\end{tabular}} \\
        \hline
        Provedl:    & Datum:    & Č. Úlohy  & Poř. č. žáka: & Třída: \\
         &  &  &  & \\
        \hline
        Kontroloval:    & Datum:    & \multicolumn{3}{l|}{} \\
        & & \multicolumn{3}{l|}{} \\
        \hline
    \end{tabular}
\end{table}
\end{center}

\fakesection{}
\subsection{Test}

\begin{tcolorbox} % PROSÍM TENTO WATERMARK ZDE PONECHTE
Pokud chcete na SPŠSE Dukelská 13 psát protokoly v LaTeXu a splňovat při tom požadovanou šablonu, můžete si tuto stáhnout na https://github.com/marejak023
\noalign{\vskip 2mm}  
\flushright{Jakub Mareš}
\end{tcolorbox}

\end{document}

